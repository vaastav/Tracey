\section{Discussion \& Future Work}
\label{sec:discussion}

\subsection{Lessons Learned}

Initially for the text generation task, we wanted to apply textual entailment techniques
when generating the text but we realized that this will result in loss of detailed information
that is useful for performing comparisons.

For the text summarization task, our plan was to use an out-of-the-box summarizer like BERT or
another state-of-the-art multi-document summarizer. But we realized quickly that an out-of-the-box
summarizer would not work well with our data as the order of the sentences would not be preserved
as well as the fact that the summarizer won't scale to a large number of traces.

\subsection{Limitations}

Our current text representation has 2 issues. The first issue is that even though the text representation
captures the temporal and causal ordering, the text representation fails to capture how much time was spent between
two pairs of events. This information is useful for diagnosing performance issues especially latency bugs.
The second issue is that the text representation of a task merely presents the information about what happened
and does not provide any sort of high level inference about the events that happened. The quality of the text representation
is useful for comparison tasks but not useful for humans as a standalone piece of information.

The summary text generated for each task flattens the graph structure by performing a topological sort.
This results in there being no indication of branching in the generated text. This is fine for performing
comparisons but according to our user study, this information is hard to digest as standalone information.

\subsection{Future Work}

As part of our future work, we would enhance the text representation of a trace to address the limitations.
First, we would introduce a special character that represents a pre-defined time unit. Then, each pair
of sentences would be separated by the number of time unit characters that represents the amount of time
elapsed between the events corresponding to the sentences.

Moreover, it is sufficiently clear, that the text representation of a trace generated for comparison is only useful
for comparison but not necessarily for understanding the execution of a trace. Thus, we would need to develop
some inference algorithm that can generate high-level inference over the text representation of the trace
to provide the user with high-level information about the trace.

We also want to enhance our summarization algorithm to somehow include the weights of the edges in the summary graph
in the generated summary for each task. This would also lead to us creating a weighted distance function whilst performing
comparisons using the aggregated summary of traces.