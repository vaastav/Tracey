\documentclass[11pt,a4paper]{article}
\usepackage[hyperref]{naaclhlt2019}
\usepackage{times}
\usepackage{latexsym}

\usepackage{graphicx}
\usepackage{subcaption}

\usepackage{listings}

% Add line breaks to urls
\usepackage{url}
\makeatletter
\g@addto@macro{\UrlBreaks}{\UrlOrds}
\makeatother

\usepackage{hyperref}

% Add a fake para command
\newcommand{\fakepara}[1]{\vspace{0mm}\noindent\textbf{#1}\quad}

% Change the caption prefix for listings to a figure
\renewcommand{\lstlistingname}{Figure}

% Change the autoref names of the various subsections
\renewcommand*\sectionautorefname{Section}
\renewcommand*\subsectionautorefname{Section}
\renewcommand*\subsubsectionautorefname{Section}
% Add an autoref name for the listing environment
\newcommand*\lstlistingautorefname{Figure}
%\newcommand*\equationautorefname{Equation}

% Make the listings and figure environments share the fucking counter
\makeatletter
\AtBeginDocument{%
  \let\c@figure\c@lstlisting
  \let\thefigure\thelstlisting
  \let\ftype@lstlisting\ftype@figure % give the floats the same precedence
}
\makeatother

% The below commands color the "Figure", "Listing", and "Table" caption prefixes
\definecolor{figurecolor}{RGB}{22,90,220}
\definecolor{citecolor}{RGB}{198,81,19}
\captionsetup[figure]{labelfont={color=figurecolor}}
\captionsetup[table]{labelfont={color=figurecolor}}
\captionsetup[lstlisting]{labelfont={color=figurecolor}}
% PDF metadata
\hypersetup{ colorlinks=true, urlcolor=black, linkcolor=figurecolor, citecolor=citecolor, pdfstartview=FitH,
        pdftitle={Aggregate-Driven Trace Visualizations for Performance Debugging},
        pdfauthor={}
        }

\aclfinalcopy % Uncomment this line for the final submission
%\def\aclpaperid{***} %  Enter the acl Paper ID here

\usepackage{tikz}
\usepackage{microtype}
\usepackage{relsize}
\usepackage{xspace}
% Command for drawing numbers inside circles
%\newcommand*\circled[1]{\tikz[baseline=(char.base)]{
%            \node[shape=circle,draw,inner sep=2pt] (char) {#1};}}
\newcommand{\circled}[1]{{\small\protect\raisebox{0.5pt}{\textcircled{\raisebox{-.2pt}{\textls[-50]{\relsize{-1.5}\phantom{0}\makebox[0pt][c]{#1}\phantom{0}}}}}}\xspace}
%%%%%%%%%%%%%%%%%%%%%%%%%%%%%%%%%%%%%%%%%%%%

\usepackage{xcolor} % for \textcolor
\usepackage{amsmath} % for \hookrightarrow

% Setting the listing environment to have linebreaks
\lstset{
  columns=fullflexible,
  frame=single, % draws a box around the listing
  breaklines=true, % breaks the lines
  postbreak=\mbox{\textcolor{red}{$\hookrightarrow$}\space} % adds an extra red arrow to indicate a line break!
}

\usepackage[normalem]{ulem} % for \sout

\usepackage{amssymb} % for math symbols

% For line break in long cells of tables
\usepackage{makecell}
\renewcommand\theadalign{bc}

%\setlength\titlebox{5cm}
% You can expand the titlebox if you need extra space
% to show all the authors. Please do not make the titlebox
% smaller than 5cm (the original size); we will check this
% in the camera-ready version and ask you to change it back.

\title{Tracey - Distributed Trace Comparison and Aggregation using NLP techniques}

\author{Vaastav Anand\\
  {\tt vaastav@cs.ubc.ca} \\\And
  Joseph Wonsil \\
  {\tt jwonsil@cs.ubc.ca} \\}

\date{}

\begin{document}

\maketitle

\begin{abstract}
    Distributed systems are widespread in usage. Yet, they continue
    to be marred by bugs. Distributed tracing is a widely adopted approach that gives 
    engineers visibility into cloud systems.
    Existing tracing tools support analysis of a
    \emph{single} request and are most useful for debugging correctness
    issues.

    However, diagnosing an issue in the processing of a request,
    requires comparing the execution of a buggy request to a non-buggy request
    or even an aggregate set of requests. Some issues even require comparing
    the behavior of two different sets of requests to identify a potential issue.
    Existing trace analysis tools either do not support these use cases or produce
    an output that is not understandable.

    To rectify this, in this paper we propose a new approach for performing trace
    comparison and aggregation. The key insight of our approach is to derive a
    text representation for each trace and then perform aggregation and comparison
    on texts. The benefit of this approach is two-fold: we can leverage text summarization
    and comparison algorithms; the output produced is text which is understandable for users.
    We present algorithms for generating a text representation from a trace, summarization
    of these text representations to generate a summary of traces, 
    and comparison of traces.
\end{abstract}

\section{Introduction}

Distributed systems are prevalent in society to the extent that billions of people either directly or 
indirectly depend on the correct functioning of a distributed system. From banking applications to social
networks, from large-scale data analytics to online video streaming, from web searches to cryptocurrencies,
most of the successful computing applications of today are powered by distributed systems.
The meteoric rise of cloud computing in the past decade has only increased our dependence on these
distributed systems in our lives.

Tasks like monitoring, root cause analysis, performance comprehension require techniques that cut across component,
system, and machine boundaries to collect, correlate, and integrate data. In the past decade, distributed tracing has emerged as an effective way to gain visibility across distributed systems~\cite{mace2015pivot,mace2018universal,fonseca2007xtrace}.  
Today distributed tracing frameworks are deployed at all major internet companies~\cite{kaldor2017canopy,sigelman2010dapper,netflixtracing}; 
notable open-source examples include OpenTelemetry~\cite{opentelemetry}, Jaeger~\cite{jaeger}, and Zipkin~\cite{zipkin}; and 
observability-focused companies offer platforms centered on analysis of traces~\cite{lightstep}.

Distributed tracing tools arose out of a need to understand the behavior of \emph{individual requests}: identifying the specific services 
invoked by a request, diagnosing problematic requests, and debugging correctness issues~\cite{fonseca2007xtrace,sigelman2010dapper,macewe}.
As a result, each trace only tells the story of a single request.
A trace represents the path of one request through the system and contains information such as the timing of requests, 
the events executed, and the nodes where these events were executed. Moreover, traces can be used
to identify slow requests and understand the difference between request executions. 

However, as distributed tracing is designed for production distributed systems, a large volume of data
is produced on a daily basis. It is humanly impossible to manually analyze each trace and draw inference
about the system as a collective. Current, analysis tools, primarily focus on visualzing a single trace
and provide little help to the user for analyzing a large amount of data. We intend to remove this burden
by automatically generating a daily summary that presents the developers with the daily highlights
across the trace data.

In this project, we propose to create an NLP tool that uses data from traces to succinctly summarize
the traces that were generated in a given day to provide an informative summary to the developer
in-charge of maintaining the health and performance of the system. We believe, generating a daily summary
would alleviate some burden from the developers in terms of the sheer volume of data they need to analyze
as well as potentially provide hints regarding the sources of performance or correctness issues.
\section{Data}
\label{sec:data}

\subsection{Trace Structure}

A trace is a Directed Acyclic Graph (DAG) of spans. A span can be thought
of as a particular task that a system performs to execute a request. The granularity of the task
is user-defined and controlled. A span can represent anything from a single function execution, 
a single thread execution, or a single operating system process comprised of multiple threads.
Spans are connected to each other by parent-child relationships. 
Each span records its timing and duration, as well as arbitrary key-value annotations provided by a developer: such as logging a span's arguments.
Within a span, developers can also add \emph{events}, which are typically \textbf{unstructured, human-annotated slog messages.}
Span annotations and events are developer-defined and vary from system to system.
Each individual trace can be very large, comprising thousands of spans and events~\cite{kaldor2017canopy, las2019sifter, shkurographdiffviz}, and production systems capture traces for millions of requests per day~\cite{kaldor2017canopy}.

Most distributed tracing frameworks represent traces using spans~\cite{jaeger,opentelemetry,sigelman2010dapper}, but some frameworks are based only on events~\cite{fonseca2007xtrace}.  
For event-based frameworks, it is straightforward to group events together into spans (e.g. events occurring in the same thread). 
In this paper we use the term \emph{task} to refer to both of these concepts. In span-based tracing frameworks a task corresponds to a span and in event-based tracing
frameworks a task corresponds to a collection of events.

\subsection{Dataset}

\begin{figure*}
    \centering%
    \begin{subfigure}{0.5\textwidth}%
        \includegraphics[width=\linewidth]{"fig/events_cdf"}%
        \caption{}%
    \end{subfigure}%
    \begin{subfigure}{0.5\textwidth}%
        \includegraphics[width=\linewidth]{"fig/tasks_cdf"}%
        \caption{}%
    \end{subfigure}%
    \caption{(a) CDF of the number of events per trace; (b) CDF of the number of tasks per trace; in the DeathStarBench trace dataset}%
    \label{fig:dataset_cdf}%
\end{figure*}

For the development and evaluation of our techniques, we use the open source deathstarbench trace dataset~\cite{anand2019deathstarbenchtraces}.
The dataset contains 22285 individual traces obtained from the DeathStarBench open-source benchmark for cloud microservices~\cite{gan2019deathstar}.
The captured traces are from 7 different API types (Register user, Follow user, Unfollow user, Composepost, Write timeline, Read timeline, Read user timeline). 
Internally,the benchmark comprises 36 microservices; each high-level API call invokes an overlapping subset of the services. In addition to datasets of regular workloads, the dataset also
contains two types of anomalous traces: one with manually triggered exceptions in the internal microservices; and one arising accidentally from a configuration error in the 
deployment causing docker containers to intermittently restart and services to be temporarily unavailable.\autoref{fig:dataset_cdf} shows the CDF of the number of events
and the number of tasks for the dataset.
\section{Tasks}
\label{sec:tasks}

In this section, we explain the user tasks that we want to accomplish and how we model
these user tasks as traditional NLP tasks. The list of tasks are as follows:

\begin{enumerate}
    \item Creating a text representation for any given trace. We model this task as a \emph{Text Generation} task
        that generates text from a data source, in this case, tracing data. Thus, this is a Data2Text task.
    \item Generating an aggregate representation of traces. We model this task as a \emph{Multi-Document Summarization} task
        where each document is the text representation generated for a trace.
    \item Comparing and explaining the differences between two traces. We model this task as a \emph{Text Comparison and Similarity} task
        where instead of comparing the raw structure of the two traces, we instead compare their corresponding text representations. The
        difference between the two traces can then be modeled as a function of the edit distance between their corresponding text representations.
    \item Comparing one trace to an aggregate set of traces. We also model this task as a \emph{Text Comparison and Similarity} task
        where we comparing the text representation of a trace with the generated summary of the aggregate set of traces using our technique described
        in Task 2. Similar to Task 3, the difference between the trace and the aggregated summary can be modeled as a function of the edit distance
        between their corresponding text representations.
\end{enumerate}
\section{Contribution}
\label{sec:contribution}

\subsection{Trace2Text}

The task of text generation from data can be divided into 
three modular interdependent tasks - (i) \emph{content planning}
defines which parts of the input fields or meaning representations
should be selected; (ii) \emph{sentence planning} determines
which selected fields are to be dealt with in each output sentence;
and (iii) \emph{surface realization} generates those sentences.
To obtain processed trace data, we use a backend server
for trace processing that we had built as part of the visualization
course last semester\footnote{The source code of the backend server is located at \url{https://github.com/vaastav/TraViz/tree/master/traviz_backend}}.
The input data for each trace includes, some overview information of the trace,
the list of events, and the list of tasks for that trace. 
For each event, the human annotated-text as well as the probability of that event occurring
is included.
For each task, a list of concurrent tasks that were happening during the task's execution
are also included. \emph{The goal is to generate a text representation of the trace
that includes information describing the execution of the trace and overview information
that would be useful for the user.}

% escapechar sets the escape character that we can use to exit into latex mode.
\begin{lstlisting}[caption={Annotated overview paragraph generated for a trace},captionpos=b,label={fig:trace_owtext}, escapechar=\#]
#\circled{1}# The trace was created on 31 May, 2019 and it took around 422 seconds to complete.
#\circled{2}# The trace was associated with the following tags: ComposePost NginxWebServer5.
#\circled{3}# The trace had 388 events, out of which 28 events had less than 25.0 chance of occurring.
#\circled{4}# The execution of the request was performed by 22 tasks. 3 tasks had latency that ranked higher than the 95th percentile of the latency distribution for the respective task. 
#\circled{5}# Task performing the operation UrlShortenHandler::UploadUrlsMongoInsert had the maximum amount of contention with 7 other tasks performing the same operation at the same time for different requests.
\end{lstlisting}

\fakepara{Content Planning} The content planner parses the input data to extract the information
needed to generate the overview text and the execution text. From the list of events, the number
of anomalous events are extracted. An anomalous event is an event that has probability lower
than a pre-defined threshold. The content planner also constructs a Directed Acyclic Graph
of the events based on the causal relationships of the events. For each task in the list of tasks,
an array of temporally ordered events are created that occurred on that task. The task with most number of
concurrent tasks is also extracted.

\fakepara{Sentence Planning} The sentence planner separates the content by planning to generate one paragraph
for each task in the trace as well as one paragraph for overview information about the trace.
The information contained inside the paragraph would be the name of the task that created this task
and labels from the events associated with the task.
The overview paragraph provides overview information about the trace such as latency, tags, and other metadata.
It also includes information about anomalous events and tasks.

\fakepara{Surface Realization} The surface realization is currently static as it is done based on
a pre-defined template. \autoref{fig:trace_owtext} shows an annotated example of the overview paragraph generated for a trace.
\circled{1} provides information about when the trace was created and what was the latency of the task.
\circled{2} mentions the tags associated with the trace.
\circled{3} contains the total number of events in the trace and the number of anomalous events. 
\circled{4} has the number of tasks in the trace and the number of tasks which have latencies that lie in the 95th percentile of the latency distributions
of their respective tasks. 
\circled{5} provides the name of the task that had the most number of concurrent tasks executing at the same time.
\autoref{fig:trace_tasktext} shows an annotated example of the paragraph generated for a task in a trace.
\circled{1} describes which task, if any, created this particular task.
\circled{2} is the name of the task.
Each sentence following \circled{2} is the human-annotated label of the corresponding event that occurred on that task.
The sentences are ordered w.r.t to the ordering of the events for that task.

% escapechar sets the escape character that we can use to exit into latex mode.
\begin{lstlisting}[caption={Annotated paragraph generated for a task in a trace},captionpos=b,label={fig:trace_tasktext}, escapechar=\#]
#\circled{1}# Task NginxWebServer created task MediaHandler::UploadMedia. 
#\circled{2}# MediaHandler::UploadMedia. 
Uploading media to compose post service. Popping client from client pool. Obtaining lock on client pool mutex. Obtained lock on client pool mutex. No client available in client pool. Creating a new client. Releasing lock on client pool mutex. Connecting to client. Pushing a client into client pool. Acquiring lock on mutex. Acquired lock on mutex. Pushing client back into client pool. Releasing lock on mutex. MediaHandler::UploadMedia complete. 
\end{lstlisting}

\subsection{Trace Summarization}

We do trace summarization at the granularity of a task. Thus, instead of summarizing the text representations of all the traces we want to summarize,
we instead summarize the text representations for each task across all traces. Thus, if a set of traces has 30 different tasks,
we perform 30 different multi-document summarizations to obtain a summary for each task. These task-specific summaries
are then concatenated together to form the summary of the traces.
The reason behind doing summarization at the granularity of a task is to ensure that there is no cross-contamination
of information between the tasks as we don't want the summarized result to have an incorrect text representation.
Additionally, the format of that summary will now match the overview generated for a
single trace. This allows us to then be able to make clean comparisons between a single trace and a group of traces using their summary.
Since each task has a relatively set behavior,
we can infer that the overview paragraphs describing it will never changes drastically during normal execution. Even though it may be repeated hundreds of times, 
there will not be an explosion in the number of unique sentences found across traces. This means that the 
summarization of a task does not have to be a complex operation.
We break down our summarization in three steps - Preprocessing, Graph Construction, and Text Conversion - that we describe below.

\fakepara{Preprocessing} In the preprocessing step, text representation for each trace is broken into text representations of each
task. Based on our text generation algorithm, text representation for a task in a trace is a paragraph.
Text representations from multiple traces are then grouped together by task. For each task, there are multiple
paragraphs that need to be summarized. Each such paragraph is represented as its own document such that
there are multiple documents available for each task.

\fakepara{Graph Construction} For each task, we construct an aggregate weighted graph from the documents associated with that task.
Each node in the graph represents one unique sentence across the documents. Uniqueness is defined not only by the content
of the sentence but also by the prefix of sentences that were before the sentence. This is done to ensure that
when the documents are being aggregated, two sentences are only merged together if they have the same preceding
sentences. Each edge represents the causal and temporal ordering between sentences. This is constructed by adding
an edge between each pair of adjacent sentences. The weight of the edge represents the number of documents in
which that edge was seen.

\fakepara{Text Conversion} To convert the graph into text we first create a topological sort ordering of the vertices
in the graph. This is to flatten the graph structure as the aggregation would have induced branches in the graph
because of deviations. We choose a topological sort ordering as it guarantees to preserve the order
between sentences from all the documents. Once we have the ordering, we simply concatenate the labels
from the nodes according to the ordering to generate the summary for that particular task.
Currently, we don't use the weights of the edges in the text generation.

\subsection{Trace Diff}

\begin{figure*}[tbh]
    \begin{equation}\label{eqn:disttrace}
    \texttt{distance} = \sum_{t \in T_1 \cup T_2} disttask(t)
    \end{equation}
    
    \begin{equation}\label{eqn:disttask}
        disttask(t)=
        \begin{cases}
            \texttt{Levenshtein}(t1, t2), \texttt{if}\: t \in T_1 \cap T_2, t1 \in T_1, t2 \in T_2 \\
            P * \texttt{numsentences}(t), \texttt{if}\: t \in (T_1 \setminus T_2) \cup (T_2 \setminus T_1).\\
        \end{cases}
    \end{equation}
    \end{figure*}

To compare two traces, or one trace and one aggregate set of traces, or two different aggregate sets of traces, we choose to compare their corresponding texts. For calculating the difference
between the two text representations, we use Google's diff-match-patch library\footnote{Library available at \url{https://github.com/google/diff-match-patch}}. For explaining our algorithm, we choose to explain the comparison
of two traces but the same algorithm applies for the other two comparison scenarios.
However, instead of just computing the diff between the two texts, we strive for a much finer granularity, and instead
compute the diff between the texts by computing the diffs between the common tasks amongst the two traces.
Tasks that are only present in either trace are automatically treated as insertions or deletions in the diff.
\autoref{fig:taskdiff} shows the output of the diff function applied to one common task in two traces.
The uncolored, normal text represents sentences (i.e. events) that are present in both traces. The struck-out,
red text represents the text that is only present in trace 1 but not in trace 2. The italic, green text represents
the text that is only present in trace 2 but not in trace 1.

% escapechar sets the escape character that we can use to exit into latex mode.
\begin{lstlisting}[caption={Text Difference for a task from 2 traces},captionpos=b,label={fig:taskdiff}, escapechar=\#]
    Task NginxWebServer created task MediaHandler::UploadMedia. MediaHandler::UploadMedia. Uploading media to compose post service. Popping client from client pool. Obtaining lock on client pool mutex. Obtained lock on client pool mutex. #\textcolor{red}{\sout{No client available in client pool. Creating a new client}}\textcolor{green}{\emph{Popping client from front of the pool}}#. Releasing lock on client pool mutex. Connecting to client. Pushing a client into client pool. Acquiring lock on mutex. Acquired lock on mutex. Pushing client back into client pool. Releasing lock on mutex. MediaHandler::UploadMedia complete
\end{lstlisting}

\fakepara{Trace Distance} Based on our diff function above, we also provide a new metric for calculating
the distance between any two traces. Let $T_1$ be the set of tasks in Trace 1 and $T_2$ be the set
of tasks in Trace 2. Then, the distance between the two traces is the sum of the distances for each task
in $T_1$ and $T_2$. \autoref{eqn:disttrace} shows the formal definition of the function for calculating
the distance between the text representations of two traces.
The disttask function for a task defines the distance calculated for this task between Trace 1 and Trace 2.
If the task is present in both the traces, then the cost is simply the Levenshtein distance between the paragraphs
for this task in the two traces. However, if the task is only present in one of the traces, then the distance is
the number of sentences in the paragraph for that task multiplied by some pre-defined task missing penalty, $P$.
Currently, $P$ is set to 10 but can be changed depending on the needs of the users. \autoref{eqn:disttask}
shows the formal definition for calculating the distance between the text representations of a task in two traces.

\section{Evaluation}
\label{sec:evaluation}

\subsection{Experimental Setup}

We perform a quantitative evaluation to evaluate
the scalability of our techniques as well as the quality
of the text, comparisons, and the summaries generated.
For the Text Generation and Trace Comparison evaluation,
all results were collected on
an Intel i7-core 3.1GHz processor machine with 32GB of RAM.
For the Trace Summarization evaluation, all results were
collected on an Intel(R) Xeon (R) CPU E5-2690 v3 @ 2.60GHz
with 56GB RAM and an NVIDIA GK210GL [Tesla K80] GPU.

In addition to the quantitative evaluation, we also
performed a qualitative evaluation by conducting
an informal user study with 1 user who is one of the 
leading experts in distributed tracing. Due to time restrictions,
the user was only able to provide feedback about the text generation
and trace comparison techniques.

\subsection{Text Generation}

\begin{figure}[tbh]
    \centering
    \includegraphics[width=\linewidth]{"fig/summary_cdf"}
    \caption{CDF of generating text for all the traces in the DeathStarBench dataset}
    \label{fig:summary_cdf}
\end{figure}

\fakepara{Quantiative Analysis} To measure the efficiency of our text generation approach, we measure the time taken
to generate the text for every trace in the DeathStarBench dataset. \autoref{fig:summary_cdf}
shows the CDF of the breakdown of the total time taken to generate the text. The time taken
to generate the text is dominated by the time taken to load the data from the backend server.
However, once the data is available, the time taken to generate the text is less than 10 milliseconds
for all traces.

\fakepara{Qualitative Analysis} To evaluate the quality of the text generated, the user mentioned the following:\textcolor{red}{``The first sentence is the interesting one here; the rest of the sentences are a bit difficult to parse.
The interesting parts of the first sentence are the comparisons to general statistics about the trace dataset.  I like the last sentence, because it's starting to push towards deriving a 
root cause for the latency (ie, contention with other tasks), and I think when it's presented as text that's a very digestable representation (vs. some sort of visual interface)."}
This suggests that the expert user believes that the overview paragraph is very useful for explaining root causes of potential problems and does so better than a potential
visual interface might. Although, we can still improve our overview by performing comparison for more statistics that a user might care about. Additionally,
the expert user also suggested \textcolor{red}{``to prune away any boring information and try to get at any root causes (or simply say that the request was normal)"}. This suggests
that text that the execution information being shown to the user is too verbose and we might need to infer some high-level information about the trace instead.

\subsection{Trace Summarization}

We explored three metrics to evaluate the summarization of traces. 
First to determine how reasonable our summarization method is we performed a scalability microbenchmark using a single task from multiple traces. 
Then we measure the quality of the summary based on the number of unique sentences it was able to capture from the original documents.
Finally, we perform a macro-scalability test by summarizing all tasks from multiple traces. For each metric we run our own algorithm as well as the potara summarization tool 
as a baseline.

\fakepara{Micro-scalability} We tested the scalability of the summarizers at a document-level. We split up the summarization to be at a task level. 
This means that each task in a given summarization job will have its own set of documents. 
To perform a micro-scalability evaluation we looked at only a single task's documents. We chose a task that had over 500 documents, each one about a paragraph long. 
We ran the summarization algorithm on an increasing number of tasks and recorded the results which can be seen in \autoref{}. Our baseline, potara, is unusable after
about 100 traces. 

\fakepara{Summary Quality} We determined our summaries should be able to capture a general idea of the ``regular'' execution state of the task being summarized. 
A task may have hundreds of sentences in its documents, but only 20 are unique. As a metric for summary quality
we checked how many of the unique sentences in a tasks corpus were present in a summary. The quality results can be found in \autoref{}. Potara is not able to capture 
the a general idea of a regular execution because at max it is using 50\% of the unique sentences in a tasks corpus of sentences. 

\fakepara{Macro-scalability} While it is important to know how a summarizer works at a document level, we are summarizing entire sets of traces. Therefore
we ran scalability experiments where an increasing number of traces were summarized. Since traces can have multiple tasks, this means that each summarization job 
in this experiment is actually running multiple times, once for each task, before it is considered complete. We ran these experiments on sets up to 1000 traces large and the 
results can be found in \autoref{}. Similar to the micro-scalability benchmarks, it is important to note the lack of scalability in potara. 

\subsection{Trace Comparison}

\begin{table}[]
    \begin{tabular}{|l|c|}
    \hline
    Trace Pair                              & Distance \\
    \hline
    \hline
    Identical Traces                        & 0.0      \\
    \hline
    Non-Error traces of same type (API)     & 258.0    \\
    \hline
    1 Error, 1 Non-Error trace of same type (API) & 1610.0  \\
    \hline
    Traces of different type (API)          & 1815.0   \\
    \hline
    \end{tabular}
    \caption{Measured Distance based on our distance function for randomly chosen pairs of traces.}
    \label{tab:comparison_dist}
\end{table}

\fakepara{Quantiative Analysis} We first measured the amount of time taken to generate the diff between the text representations
of two traces. From the DeathStarBench, we randomly chose 100 different pairs of traces and measured the time taken
to generate the diff between the trace. On average, it took 4.87 milliseconds to generate the diff and measure the distance
between two pairs of traces.

\fakepara{Accuracy of Distance function} We wanted to ensure that the distance function we had come up behaves in the expected way
when computing the distance between pairs of traces. \autoref{tab:comparison_dist} shows our detailed results. First,
the distance function returns 0 when computing the distance between two identical traces. Furthermore, the distance function
grows monotonically with increase in deviation between traces. 

\fakepara{Qualitative Analysis of Diff} Regarding the quality of the comparison generated, the user mentioned the following:\textcolor{red}{``This is really cool, I'm actually 
surprised how amenable the complex trace data is to being represented as text.  I've always wondered how to visually compare two traces;I 
really like how you've leveraged some of the more established visual indicators of text comparison (the green+ / red- idiom). With trace diff,
I understand now why having a more verbose trace representation in text is useful.  It's not interesting by itself, but the diffs provide 
context for honing in on specific parts of the text.''} This suggests that the trace diff that we generated using text diff is successful.
However, we do believe that it has certain limitations that we discuss in \autoref{sec:discussion}.

\section{Discussion \& Future Work}
\label{sec:discussion}
\section{Related Work}

Jaeger ~\cite{Jaeger} is an open-source distributed tracing project
that provides libraries for instrumenting distributed systems
in different languages as well as a frontend for viewing the traces
produced from these systems. Jaeger has vis idioms for visualizing
a single trace, for visualizing the dependencies between services
of the system, as well as an idiom for comparing 2 different traces.
However, it doesn't have any idiom for viewing an aggregate form of
multiple traces or for comparing groups of traces.

LightStep ~\cite{LightStep} is a start-up company that
creates solutions for real-time tracking of requests and metrics
in large-scale distributed systems. LightStep has vis idioms
for visualizing the structure of a single trace, for visualizing
the critical path of a single trace, and 
for visualizing various metrics collected for multiple traces.
Like Jaeger, LightStep also lacks a viz idiom for viewing aggregate
traces and for comparing groups of traces.

ShiViz ~\cite{ShiViz} is an interactive visualization tool
that visualizes communication graphs from distributed system execution
logs. As ShiViz is designed for visualizing logs and not individual traces,
ShiViz does not support comparisons of multiple traces.

With TraViz, we want to address some of the shortcomings of existing tools
by providing a way for the users to compare the structure of a group of traces
as well as integrate the traces with source code to provide more context
for debugging.
\section{Conclusion}
\label{sec:conclusion}

\bibliography{report}
\bibliographystyle{acl_natbib}

\appendix

\end{document}
