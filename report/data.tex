\section{Data}
\label{sec:data}

\subsection{Trace Structure}

A trace is a Directed Acyclic Graph (DAG) of spans. A span can be thought
of as a particular task that a system performs to execute a request. The granularity of the task
is user-defined and controlled. A span can represent anything from a single function execution, 
a single thread execution, or a single operating system process comprised of multiple threads.
Spans are connected to each other by parent-child relationships. 
Each span records its timing and duration, as well as arbitrary key-value annotations provided by a developer: such as logging a span's arguments.
Within a span, developers can also add \emph{events}, which are typically \textbf{unstructured, human-annotated slog messages.}
Span annotations and events are developer-defined and vary from system to system.
Each individual trace can be very large, comprising thousands of spans and events~\cite{kaldor2017canopy, las2019sifter, shkurographdiffviz}, and production systems capture traces for millions of requests per day~\cite{kaldor2017canopy}.

Most distributed tracing frameworks represent traces using spans~\cite{jaeger,opentelemetry,sigelman2010dapper}, but some frameworks are based only on events~\cite{fonseca2007xtrace}.  
For event-based frameworks, it is straightforward to group events together into spans (e.g. events occurring in the same thread). 
In this paper we use the term \emph{task} to refer to both of these concepts. In span-based tracing frameworks a task corresponds to a span and in event-based tracing
frameworks a task corresponds to a collection of events.

\subsection{Dataset}

\begin{figure*}
    \centering%
    \begin{subfigure}{0.5\textwidth}%
        \includegraphics[width=\linewidth]{"fig/events_cdf"}%
        \caption{}%
    \end{subfigure}%
    \begin{subfigure}{0.5\textwidth}%
        \includegraphics[width=\linewidth]{"fig/tasks_cdf"}%
        \caption{}%
    \end{subfigure}%
    \caption{(a) CDF of the number of events per trace; (b) CDF of the number of tasks per trace; in the DeathStarBench trace dataset}%
    \label{fig:dataset_cdf}%
\end{figure*}

For the development and evaluation of our techniques, we use the open source deathstarbench trace dataset~\cite{anand2019deathstarbenchtraces}.
The dataset contains 22285 individual traces obtained from the DeathStarBench open-source benchmark for cloud microservices~\cite{gan2019deathstar}.
The captured traces are from 7 different API types (Register user, Follow user, Unfollow user, Composepost, Write timeline, Read timeline, Read user timeline). 
Internally,the benchmark comprises 36 microservices; each high-level API call invokes an overlapping subset of the services. In addition to datasets of regular workloads, the dataset also
contains two types of anomalous traces: one with manually triggered exceptions in the internal microservices; and one arising accidentally from a configuration error in the 
deployment causing docker containers to intermittently restart and services to be temporarily unavailable.\autoref{fig:dataset_cdf} shows the CDF of the number of events
and the number of tasks for the dataset.