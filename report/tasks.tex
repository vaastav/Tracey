\section{User Tasks}
\label{sec:tasks}

In this section, we explain the user tasks that we want to accomplish and how we model
these user tasks as traditional NLP tasks. The list of tasks are as follows:

\begin{enumerate}
    \item Creating a text representation for any given trace. We model this task as a \emph{Text Generation} task
        that generates text from a data source, in this case, tracing data. Thus, this is a Data2Text task.
    \item Generating an aggregate representation of traces. We model this task as a \emph{Multi-Document Summarization} task
        where each document is the text representation generated for a task from a single trace.
    \item Comparing and explaining the differences between two traces. We model this task as a \emph{Text Similarity} task
        where instead of comparing the raw structure of the two traces, we instead compare their corresponding text representations. The
        difference between the two traces can then be modeled as a function of the edit distance between their corresponding text representations.
    \item Comparing one trace to an aggregate set of traces. We also model this task as a \emph{Text Similarity} task
        where we compare the text representation of a trace with the generated summary of the aggregate set of traces using our technique described
        in Task 2. Similar to Task 3, the difference between the trace and the aggregated summary can be modeled as a function of the edit distance
        between their corresponding text representations.
\end{enumerate}