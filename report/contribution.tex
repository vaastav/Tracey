\section{Contribution}
\label{sec:contribution}

\subsection{Trace2Text}

The task of text generation from data can be divided into 
three modular interdependent tasks - (i) \emph{content planning}
defines which parts of the input fields or meaning representations
should be selected; (ii) \emph{sentence planning} determines
which selected fields are to be dealt with in each output sentence;
and (iii) \emph{surface realization} generates those sentences.
To obtain processed trace data, we use a backend server
for trace processing that we had built as part of the visualization
course last semester\footnote{The source code of the backend server is located at \url{https://github.com/vaastav/TraViz/tree/master/traviz_backend}}.
The input data for each trace includes, some overview information of the trace,
the list of events, and the list of tasks for that trace. 
For each event, the human annotated-text as well as the probability of that event occurring
is included.
For each task, a list of concurrent tasks that were happening during the task's execution
are also included. \emph{The goal is to generate a text representation of the trace
that includes information describing the execution of the trace and overview information
that would be useful for the user.}

% escapechar sets the escape character that we can use to exit into latex mode.
\begin{lstlisting}[caption={Annotated overview paragraph generated for a trace},captionpos=b,label={fig:trace_owtext}, escapechar=\#]
#\circled{1}# The trace was created on 31 May, 2019 and it took around 422 seconds to complete.
#\circled{2}# The trace was associated with the following tags: ComposePost NginxWebServer5.
#\circled{3}# The trace had 388 events, out of which 28 events had less than 25.0 chance of occurring.
#\circled{4}# The execution of the request was performed by 22 tasks. 3 tasks had latency that ranked higher than the 95th percentile of the latency distribution for the respective task. 
#\circled{5}# Task performing the operation UrlShortenHandler::UploadUrlsMongoInsert had the maximum amount of contention with 7 other tasks performing the same operation at the same time for different requests.
\end{lstlisting}

\fakepara{Content Planning} The content planner parses the input data to extract the information
needed to generate the overview text and the execution text. From the list of events, the number
of anomalous events are extracted. An anomalous event is an event that has probability lower
than a pre-defined threshold. The content planner also constructs a Directed Acyclic Graph
of the events based on the causal relationships of the events. For each task in the list of tasks,
an array of temporally ordered events are created that occurred on that task. The task with most number of
concurrent tasks is also extracted.

\fakepara{Sentence Planning} The sentence planner separates the content by planning to generate one paragraph
for each task in the trace as well as one paragraph for overview information about the trace.
The information contained inside the paragraph would be the name of the task that created this task
and labels from the events associated with the task.
The overview paragraph provides overview information about the trace such as latency, tags, and other metadata.
It also includes information about anomalous events and tasks.

\fakepara{Surface Realization} The surface realization is currently static as it is done based on
a pre-defined template. \autoref{fig:trace_owtext} shows an annotated example of the overview paragraph generated for a trace.
\circled{1} provides information about when the trace was created and what was the latency of the task.
\circled{2} mentions the tags associated with the trace.
\circled{3} contains the total number of events in the trace and the number of anomalous events. 
\circled{4} has the number of tasks in the trace and the number of tasks which have latencies that lie in the 95th percentile of the latency distributions
of their respective tasks. 
\circled{5} provides the name of the task that had the most number of concurrent tasks executing at the same time.
\autoref{fig:trace_tasktext} shows an annotated example of the paragraph generated for a task in a trace.
\circled{1} describes which task, if any, created this particular task.
\circled{2} is the name of the task.
Each sentence following \circled{2} is the human-annotated label of the corresponding event that occurred on that task.
The sentences are ordered w.r.t to the ordering of the events for that task.

% escapechar sets the escape character that we can use to exit into latex mode.
\begin{lstlisting}[caption={Annotated paragraph generated for a task in a trace},captionpos=b,label={fig:trace_owtext}, escapechar=\#]
#\circled{1}# Task NginxWebServer created task MediaHandler::UploadMedia. 
#\circled{2}# MediaHandler::UploadMedia. 
Uploading media to compose post service. Popping client from client pool. Obtaining lock on client pool mutex. Obtained lock on client pool mutex. No client available in client pool. Creating a new client. Releasing lock on client pool mutex. Connecting to client. Pushing a client into client pool. Acquiring lock on mutex. Acquired lock on mutex. Pushing client back into client pool. Releasing lock on mutex. MediaHandler::UploadMedia complete. 
\end{lstlisting}

\subsection{Trace Summarization}

Trace summarization is done at the granularity of a task. This means that all of the documents generated in the previous steps undergo another round of processing.
Looking at the task level gives us a clear way to create summaries of a group of traces. Additionally, the format of that summary will match the overview generated for a
single trace. This allows us to then be able to make clean comparisons between a single trace and a group of traces using their summary.
us to accurately evaluate a summary as well as provide a simple method to compare a single trace to a set of traces. Since each task has a relatively set behavior,
we can infer that the overview paragraphs describing it will never changes drastically during normal execution. Even though it may be repeated hundreds of times, 
there will not be an explosion in the number of unique sentences found across traces. This means that the 
summarization of a task does not have to be a complex job. In fact the simplest method would be taking the set of all unique sentences in the task found across
traces. However, this would lose the temporal and casual information inherent in the order of the sentences. 

To preserve this information we provide a graph-based
approach. Each document in a task is analyzed and the sentences are mapped into a directed graph. Each node in the graph is a unique sentence, and the edges are
weighted based on how often the connecting node was the following sentence in a document. 

\subsection{Trace Diff}