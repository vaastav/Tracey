\section{Related Work}
\label{sec:related}

Building trace aggregation and trace comparison tools have garnered a lot of effort from both industry and academia in the past decade.
All of these efforts have been focused at developing visualization techniques.
Pintrace~\cite{pintrace}, the distributed tracing system at Pinterest, uses aggregate analysis to compare two different groups of traces to 
narrow down the root cause of an error. However, the granularity of the comparison is coarse as the comparison
is only performed on the distribution of latencies of the traces in each group.
This is only useful for identifying high-level trends. Early work on trace comparison similarly compared latency distributions,
but required traces to be structurally isomorphic~\cite{sambasivan2011diagnosing, sambasivan2013visualizing}; this is rare in practice as
most traces are structurally unique~\cite{las2019sifter}. XTrace~\cite{fonseca2007xtrace} has a trace comparison tool
that uses a graph difference algorithm to visualize the diff between the event graphs of two traces. Although, the visualization
can highlight unique events in each trace, it fails to explain to the user what these events are.
Recent work at Uber~\cite{shkurographdiffviz} compares the structure of incoming 
traces with a ``good'' set of traces to find ``bad'' traces and then visualizes their difference as a directed graph. This involves 
extracting features from the ``good'' traces and then comparing that against a potentially ``bad'' trace. It is not particularly
clear how effective such a visualization is in practice. The techniques we have discussed use textual representation instead of visual
representation of traces which we believe is superior for explaining the differences between traces as well as similarities between traces.

Traditional text generation systems generate text from data using hand-crafted specifications and rules for generating
text from data~\cite{dale2003coral,reiter2005choosing,portet2009automatic,turner2009generating}. However,
there have also been a rise in data-driven and neural approaches that aim to learn the working of the text generation
modules - content planning, sentence planning, and surface realization - either individually or in a combined fashion~\cite{barzilay2005collective,barzilay2006aggregation,konstas2013global,lebret2016neural}.
Our text generation approach is similar to the traditional approach where we write explicit templates and rules
for generating text from a trace. We leave the use of neural and data-driven approaches for text generation from
traces as future work.

Text similarity is a widely-studied task with many different techniques and measures~\cite{gomaa2013survey}.
Similarity measures are broadly classified into two categories - string-based similarity and corpus-based similarity.
In this work, we choose Levenshtein distance as our similarity measure for trace text comparison as Levenshtein distance
accounts for insertions, deletions, and substitutions which corresponds one-to-one with missing events, new events,
and reordered events. It additionally has the benefit of accounting for the ordering of the letters/words in the strings
which is important for our trace text as the ordering of sentences in the strings for a task inside a trace's text
encodes temporal relationship between the events.

Multi-Document Summarization techniques are either extractive or abstractive. Extractive techniques~\cite{xiao2019extractive,liu2019hierarchical,erkan2004lexrank} extract key sentences
and phrases from documents based on some sort of similarity score between pairs of phrases or sentences and
uses a ranking algorithm as to select the phrases to be included in the summary. However, extractive techniques do not respect
the ordering of the sentences which are temporally relevant for the trace text we are summarizing.
Abstractive techniques~\cite{devlin2018bert,bing2015abstractive} formulate facts from the documents and then attempts to rewrite the summary from scratch. These approaches
do not fit our end-goal of using the aggregate trace text summary for comparison with other traces. However, we do note
that these techniques can be useful for generating a summary that provides a better explanation of the traces to users.