\section{Evaluation}

We have identified three metrics we will evaluate for our tool. They are generation speed, summarization ability,
and preservation of important information. These metrics were chosen to make sure our tool will be able to provide 
important data to users within a reasonable time. 

We will measure generation speed by creating summaries of groups of traces that vary in size. We will compare the time it took 
to generate the summaries to the number of traces in each group. For summaries generated from natural language queries the time
should take no longer than a few seconds. For daily reports, these will likely be generated overnight and could take longer, 
up to a few minutes. 

We will measure the summarization ability of our tool by comparing lengths of summaries to the number of traces that were involved
their generation. As the number of traces increase, the lengths of the summaries should not grow exponentially or linearly. We want the 
reports to be no longer than a page for large numbers of traces and a paragraph for smaller nubmers. Additionally, we will perform measures 
of redundancy to ensure we are not repeating information in our summaries, making them artifically long. We will obtain these measures using
a unique n-gram ratio and normalized inverse of diversity.

Finally we will measure preservation of important information by manually identifying critical data in the traces, and counting how many
appear in summarizations. This evaluation will be applied to traces of varying sizes. We expect to preserve the majority of critical 
information for daily reports; however, understand that as the number of traces increase we could lose some if we choose to cap the 
summary size. 